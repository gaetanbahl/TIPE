\documentclass{article}
\usepackage[T1]{fontenc}
\usepackage[francais]{babel}     
\usepackage[utf8]{inputenc}
\usepackage{graphicx}
\title{TIPE: Tic-Tac-Toe et les arbres de décision}
\author{Gaétan Bahl et Quentin Adam}

\begin{document}
\maketitle
\tableofcontents


\clearpage

\section{Introduction aux arbres}

Les arbres de décisions sont nés en 1963 alors que Morgan et Sonquist se sont attaqués au problème de prédiction de variables. Bien que la précision de chaque probabilité et le nombre de branches aient augmenté, la manière de penser les arbres de décisions n'a pas changé pour autant. En effet, l'algorithme recherche toujours la branche avec le poids le plus important. C'est donc pourquoi les intelligences artificielles ont pris le relais.
En s'intéressant aux intelligences artificielles et surtout aux moyens que
celles-ci utilisent pour prendre leur décisions
on est vite confronté dans le cas des jeux (de société, par exemple), à la
représentation de données par des arbres dans leur sens mathématique ou
informatique.


\subsection{Qu'est-ce qu'un arbre ?}

Un arbre est une donnée représentant un ensemble $E$ et une relation binaire $R$
symétrique qui lie les éléments de cet ensemble,
de sorte que deux éléments de l'ensemble soient liés entre eux par un unique
chemin le plus court possible.






\section{\uppercase{à} quoi servent les arbres de décision ?}

\subsection{En automatique}

Les arbres de décision sont largement utilisés par les automates,
 qui doivent réagir à un nombre fini de situations possibles, 
leur comportement est donc géré par des arbres qui contiennent
 toutes les états dans lesquels peuvent se trouver les différents 
éléments avec lesquels doit intéragir l'automate.
Dans ce cas là, les arbres sont souvent binaires, 
c'est à dire qu'ils se résument à <<Si telle situation arrive>>, alors <<faire
cela>>.

\subsection{En entreprise}

Les décisions possibles dans le cas de l'économie sont aussi prises grâce à des
arbres de décision,
on représente les différentes solutions à un problème avec leurs conséquences,
puis on choisit 
la marche à suivre en fonction de ça.

\subsection{Dans l'industrie}

Les ouvriers qui travaillent dans l'industrie ont un <<process>> écrit 
leur indiquant la marche à suivre en cas de problème, 
ou au contraire si tout se passe bien. Par exemple dans l'imprimerie, il y a une
méthode à executer si la 
couleur de sortie n'est pas exactement la bonne.
Cela revient par exemple à <<si le mélange n'est pas bon, ajouter un peu de
teinte à l'encre>> en boucle.


\subsection{Dans les intelligences artificielles}

Les intelligences artificielles, ou IA sont souvent basées sur des abres de
décision 

\pagebreak

\section{Les arbres dans les jeux et application au <<Tic tac Toe>>}

Parmi les utilisations des arbres de décisions, on trouve les arbres de jeu.
Les arbres sont pratiques dans le cas de la création d'une intelligence
artificielle puisque 
ceux-ci sont adaptés à l'exploration
de tous les resultats et issues possibles à un certain moment du jeu,
l'intelligence artificielle évalue ensuite quelle partie de l'arbre sera la plus
propice à une victoire et choisit son coup en fonction de cela. C'est ce qui est
utilisé notamment pour la création d'intelligences
artificielles pour le jeu d'échecs, du moins, en partie, puisqu'il est assez compliqué de créer
 un arbres de toutes les issues possibles d'une partie d'échecs.

\subsection{Introduction au jeu <<Tic Tac Toe>>}

Le jeu <<Tic tac Toe>>, aussi appelé <<morpion>> est un jeu simple, à deux joueurs, se jouant au tour à tour sur une grille de 3*3 cases.
Le joueur qui commence la partie et que l'on nommera <<joueur 1>> a le rôle des $O$, l'autre joueur a le rôle des $X$, à l'origine.
Le but du joueur est d'aligner 3 de ses signes, que ce soit en ligne, en colonne ou en diagonale dans la grille.

Dans ce jeu, deux joueurs expérimentés ou deux intelligences artificielles qui jouent ensemble arrivent toujours à un match
nul, on parle de jeu \emph{futile}.

%\marginpar{\includegraphics[scale=0.5]{image1.eps}\\ Morpion}

\subsection{L'algorithme Min-Max}

L'exploitation des arbres de jeu peut être faite par l'algorithme Min-Max 
que nous allons appliquer au jeu du <<Tic Tac Toe>>. \\

Par définition, l'algorithme Min-
Max se prête à un jeu à deux joueurs dans le but d'appliquer une stratégie
gagnante. 
Afin d'introduire cet algorithme, expliquons la condition de somme nulle qui
doit être remplie pour Min-Max : 
lorsque le joueur gagne une partie, on ajoute 1 à son gain et respectivement -1
lors 
d'une défaite et 0 lors d'une partie nulle. \\

On peut donc considérer que le <<Tic Tac Toe>> est un jeu à somme nulle.
Dans la suite on se place dans le cadre du <<Tic Tac Toe>>. \\

L'algorithme Min-Max est donc simple : après que l'IA ai calculé toutes les
possibilités de jouer,
 il choisi la meilleure option pour gagner. \\

On s'imagine alors bien que Min-Max peut calculer à différents ordres. L'ordre 1
correspond à l'ordre auquel 
l'IA calcule uniquement les différentes possibilités pour le coup suivant et
Min-Max cherche uniquement 
la meilleure solution pour le coup suivant. De même, on définit donc l'ordre 2
et plus.
Donc, à l'ordre 2, Min-Max évalue son prochain coup et le coup de l'adversaire,
et évidemment 
l'IA considère que les deux coups joués sont effectués en vue de gagner. L'IA
"veut" donc 
maximiser son score sur le prochain coup, tout en minimisant le coup suivant de
l'adversaire.
D'où le terme de Min-Max.


\section{Présentation des programmes}

Dans cette partie, nous allons exposer les différentes parties du programme permettant d'évaluer le poids de chaque branche et donc de déterminer la meilleure option pour gagner. Le langage de programmation utilisé est Python.

\subsection{La génération de l'arbre}

Une structure personnelle à base de dictionnaires a été utilisée pour la mise en mémoire des données. Les noeuds de l'arbre sont représentés par les grilles de jeu à un instant de la partie, les branches qui en découlent sont toutes les possibilités de coup envisageables. L'arbre est construit récursivement à partir la grille de départ par la fonction \emph{arbre_a_partir_de}, qui va explorer toutes les possibilités. Le temps de calcul n'est pas excessivement long car finalement, le jeu de morpion est très simple et il n'y a <<qu'environ>> 200 000 parties possibles. C'est une solution de génération qui ne serait pas envisageable au jeu d'échecs par exemple, car le temps de calcul serait beaucoup trop long (il y a 2x10\up{116} parties possibles aux échecs), en effet, il faudrait 10\up{21} siècles au plus puissant calculateur d'échecs pour établir tout l'arbre.

\subsection{Les calculs statistiques}

Bien que ce ne soit pas exactement le sujet ici, pour vérifier que l'arbre était bien juste et que notre algorithme était fonctionnel, nous avons créé des fonctions permettant de trouver les nombres de parties possibles et de parties gagnantes possibles pour chaque coup, chiffres que nous avons comparés à ceux que l'on peut trouver par dénombrement.

%insérer le calcul de dénombrement ici

Les chiffres nous permettent de dire que le joueur qui commence la partie a un avantage considérable quant à l'issue de la partie, puisqu'il y a quasiment deux fois plus de combinaisons favorables à sa victoire. On constate aussi que si le joueur qui commence la partie joue dans un des coins de la grille, ses <<chances>> de gagner sont quasiment doublées.






\end{document}
