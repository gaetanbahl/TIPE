\documentclass{article}
\usepackage[T1]{fontenc}
\usepackage[francais]{babel}     
\usepackage[utf8]{inputenc}

\title{TIPE: Tic-Tac-Toe et les arbres de décision}
\author{Gaétan Bahl et Quentin Adam}

\begin{document}
\maketitle
\tableofcontents

\section{\uppercase{à} quoi servent les arbres de décision ?}

\subsection{En automatique}

Les arbres de décision sont largement utilisés par les automates,
 qui doivent réagir à un nombre fini de situations possibles, 
leur comportement est donc géré par des arbres qui contiennent
 toutes les états dans lesquels peuvent se trouver les différents 
éléments avec lesquels doit intéragir l'automate.
Dans ce cas là, les arbres sont souvent binaires, 
c'est à dire qu'ils se résument à <<Si telle situation arrive>>, alors <<faire cela>>.

\subsection{En entreprise}

Les décisions possibles dans le cas de l'économie sont aussi prises grâce à des arbres de décision,
on représente les différentes solutions à un problème avec leurs conséquences, puis on choisit 
la marche à suivre en fonction de ça.

\subsection{Dans l'industrie}

Les ouvriers qui travaillent dans l'industrie ont un <<process>> écrit 
leur indiquant la marche à suivre en cas de problème, 
ou au contraire si tout se passe bien. Par exemple dans l'imprimerie, il y a une méthode à executer si la 
couleur de sortie n'est pas exactement la bonne.
Cela revient par exemple à <<si le mélange n'est pas bon, ajouter un peu de teinte à l'encre>> en boucle.

\subsection{Dans les intelligences artificielles}

Les intelligences artificielles, ou IA sont souvent basées sur des abres de décision 

\end{document}
